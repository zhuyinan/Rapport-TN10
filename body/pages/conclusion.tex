\chapter{Conclusion}
%\addcontentsline{toc}{section}{Conclusion}

Lorsque de mon stage de fin d'étude de 6 mois, j'ai appris beaucoup de chose.

Tout d'abord, j'arrive de intégration dans une équipe rapidement. Auparavant, à cause de problème de mon niveau de langue, ça prend du temps de bien comprendre le sujet et la fonctionnalité du groupe. Par conséquence, ce n'est pas pratique de intégrer dans un groupe de développement. 

Grâce à l'augmentation de niveau de mon langue français  et aussi beaucoup de méthodologie j'ai appris pendant ma formation dans filière SRI à UTC, j'arrive de intégrer dans le développement rapidement après un mois de pré études chez Data-Gest, et aussi à l'aide de mon tuteur qui m'a expliqué la  fonctionnalité et structure de framework en détail avec patience. 

Deuxièmement, j'arrive de  mettre en pratique mes connaissances théoriques acquises durant mes études à UTC . Surtout dans la partie de l'administration de système et configuration du réseaux. Après 6 mois de m'entraîner, je peux bien maîtriser VIM, et aussi beaucoup de méthodes techniques d'utiliser BASH commandes sous Linux.

Troisième, mon niveau de programmation est augmenté. Surtout dans la programmation orienté objet. Au début, j'avais juste un concept de POO, après avoir pratiqué 6 mois pendant mon stage, ma connaissance de POO est enrichi.

Dernièrement, j'ai bien pratiqué la méthodologie de la gestion du projet. Pendant mon stage, il faut souvent travailler sur plusieurs sujet simultanément. En utilisant la méthodologie, j'arrive de coordonner des différents projets. 

Le stage fin d'étude est un excellent souvenir, il constitue désormais une expérience professionnelle valorisante et encourageante pour mon carrière future. J'ai vécu une expérience enrichissante de travail pour la suite et mes futurs emplois. 

En résumé, je trouve que ce stage a été très bénéfique. Et aussi, je tiens à exprimer ma satisfaction d'avoir pu travaillé dans de bonnes conditions matérielles et un environnement agréable


